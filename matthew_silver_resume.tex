\documentclass{report}
\usepackage{ms_resume_style}

\begin{document}


  %
  % Matthew R. Silver
  %
  % ===================================================================================

  \begin{resume_header}{Matthew R. Silver, Ph.D.}{matthew.r.silver@gmail.com}{\directlua{tex.print(phone)}}{matthewsilver}{matthewrsilver}

    Data scientist with a background in computational neuroscience emphasizing neural networks, machine learning, dynamical systems modeling, and data analysis. Over twenty years of programming experience, and formal education in computer science. Excellent communicator and team player eager to work with talented people to solve complex, real world problems.

  \end{resume_header}



  %
  % Technical Skills
  %

  \sectionheader{Technical Skills}

  \begin{skillset}{Machine Learning and Neural Networks}
    \item Extensively familiar with feedforward and recurrent neural networks; knowledgeable about deep learning and LSTM
    \item Theoretical understanding and practical experience applying supervised and unsupervised machine learning techniques
    \item Knowledgeable about anomaly detection, with experience utilizing a variety of techniques in the network security space
    \item Skilled practitioner of critical data handling methods: from dimensionality reduction to cross validation to visualization
  \end{skillset}

  \begin{skillset}{Programming Languages, Frameworks, and Software}
    \item Proficient with Python (tensorflow, numpy, scikit, networkx, pandas); experience with Scala, R, JavaScript, Bash, Matlab
    \item Strong \CC{} engineering experience; system-level thinker who writes sound, performant production code
    \item Experienced working with both structured data (SQL) and semi-structured data (JSON, CouchDB, HDFS/Spark)
  \end{skillset}



  %
  % Work Experience
  %

  \sectionheader{Work Experience}


  % Vectra Networks

  \begin{work_location}{Vectra Networks}{October 2014}{present}

    \begin{position}{Lead Data Scientist}
      \item \looseness=-1 Managed and built a team of five data scientists; coordinated with HQ to found and grow Vectra's Cambridge, MA office
      \item Planned the algorithm development roadmap for the data science team in collaboration with product and security teams
      \item Designed and implemented a framework for detecting and communicating ongoing attack campaigns to customers
    \end{position}

    \begin{position}{Senior Data Scientist}
      \item Promoted from Data Scientist to Senior Data Scientist after 11 months, in August 2015
      \item Created deep recurrent neural networks for learning embeddings of network time series data, which resulted in a patent
      \item Implemented and improved upon algorithms in \CC{} and Python for detecting cyber attacks based on information flow
      \item Crafted various anomaly detection systems, including one based on a deep autoencoder using \HzO{}, R and Scala
      \item Developed a system to model hosts in customer networks and identify when the host utilizing an address has changed
    \end{position}

  \end{work_location}


  % Massachusetts Institute of Technology

  \begin{work_location}{Massachusetts Institute of Technology}{2011}{2014}

    \begin{position}{Postdoctoral Associate, Department of Brain and Cognitive Sciences}
      \item \looseness=-1 Performed electrophysiological experiments: recorded signals in multiple brain areas as monkeys performed cognitive tasks
      \item Analyzed neural data with Matlab to examine system-level interactions between different brain areas
      \item Developed training and analysis infrastructure in Matlab to control and evaluate daily behavior of animals
      \item Supervised graduate students as they developed experiments, built neural recording setups, and ran analyses
    \end{position}

  \end{work_location}


  % Boston University

  \begin{work_location}{Boston University}{2005}{2011}

    \begin{position}{Research Assistant \& Teaching Assistant, Department of Cognitive and Neural Systems}
      \item Explored and implemented neural networks and machine learning algorithms while studying their theoretical foundations
      \item Designed a mathematical model of working memory using interacting systems of differential equations
      \item Implemented the model in Matlab and \CC{} to simulate brain function and explain experimental data
    \end{position}

  \end{work_location}


  %
  % Patents
  %

  \sectionheader{Patents}

  \patent {Silver, Matthew R. and Kazerounian, Sohrob}{2017}{Method and System for Learning Representations of Network Flow Traffic}{US Patent Number 12345}{January 27, 2018}

  %
  % Education
  %

  \sectionheader{Education}

  \degree{Ph.D., Cognitive and Neural Systems (Computational Neuroscience)}{Boston University, Boston, MA}{2005}{2011}

  \degree{B.A., Neuroscience, minor concentration in Computer Science}{Hamilton College, Clinton, NY}{2001}{2005}


\end{document}
